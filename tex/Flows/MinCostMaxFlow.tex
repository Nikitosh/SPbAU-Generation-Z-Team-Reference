\cpibegin
const int MAX_M = 1e4;
int pr[MAX_N], in[MAX_N], q[MAX_N * MAX_M], used[MAX_N], d[MAX_N], pot[MAX_N];
vi g[MAX_N];

struct Edge {
	int v, u, c, f, w;
	Edge() {}
	Edge(int v, int u, int c, int w): v(v), u(u), c(c), f(0), w(w) {}
};

vector <Edge> edges;

inline void addFlow(int e, int flow) {
 	edges[e].f += flow;
 	edges[e ^ 1].f -= flow;
}

inline void addEdge(int v, int u, int c, int w) {
	g[v].pb(sz(edges));
	edges.pb(Edge(v, u, c, w));
	g[u].pb(sz(edges));
	edges.pb(Edge(u, v, 0, -w));
}

void read(int m) {
 	forn (i, m) {
 	 	int v, u, c, w;
 	 	scanf("%d%d%d%d", &v, &u, &c, &w);
 	 	addEdge(v - 1, u - 1, c, w);
 	}
}

int dijkstra(int n, int s, int t) {
	forn (i, n)
		used[i] = 0, d[i] = INF;
	d[s] = 0;
	while (1) {
	 	int v = -1;
	 	forn (i, n)
	 		if (!used[i] && (v == -1 || d[v] > d[i]))
	 			v = i;
	 	if (v == -1 || d[v] == INF)
	 		break;
	   	used[v] = 1;
	 	for (int edge : g[v]) {
	 	 	auto &e = edges[edge];
	 	 	ll w = e.w + pot[v] - pot[e.u];
	 	 	if (e.c > e.f && d[e.u] > d[v] + w)
	 	 		d[e.u] = d[v] + w, pr[e.u] = edge;	
	 	}
	}
	if (d[t] == INF)
		return d[t];
	forn (i, n)
		pot[i] += d[i];
	return pot[t];
}

int fordBellman(int n, int s, int t) {
	forn (i, n)		
		d[i] = INF;
	int head = 0, tail = 0;
	d[s] = 0;
	q[tail++] = s;
	in[s] = 1;
	while (tail - head > 0) {
		int v = q[head++];
		in[v] = 0;
		for (int edge : g[v]) {
			auto &e = edges[edge];
			if (e.c > e.f && d[e.u] > d[v] + e.w) {
				d[e.u] = d[v] + e.w;
				pr[e.u] = edge;
				if (!in[e.u])
					in[e.u] = 1, q[tail++] = e.u;	
			} 
		}
	}
	return d[t];
}

int minCostMaxFlow(int n, int m, int s, int t) {
	read(m);
	int ansFlow = 0, ansCost = 0, dist;
	while((dist = dijkstra(n, s, t)) != INF) {
		int curFlow = INF;
		for (int cur = t; cur != s; cur = edges[pr[cur]].v)
			curFlow = min(curFlow, edges[pr[cur]].c - edges[pr[cur]].f); 
		for (int cur = t; cur != s; cur = edges[pr[cur]].v)
			addFlow(pr[cur], curFlow);
		ansFlow += curFlow;
		ansCost += curFlow * dist;
	}
	return ansCost;
}
\end{minted}
   