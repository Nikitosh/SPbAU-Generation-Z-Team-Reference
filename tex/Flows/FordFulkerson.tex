\cpibegin
int used[MAX_N], pr[MAX_N];
vi g[MAX_N];
int curTime = 1;

#include "Utilities.cpp"

int dfs(int v, int can, int toPush, int t) {
	if (v == t)
		return can;
	used[v] = curTime;
	for (int edge : g[v]) {
		auto &e = edges[edge];
		if (used[e.u] != curTime && e.c - e.f >= toPush) {
			int flow = dfs(e.u, min(can, e.c - e.f), toPush, t);
			if (flow > 0) {
				addFlow(edge, flow);
				pr[e.u] = edge;
				return flow;
			}
		}
	}
	return 0;
}

int fordFulkerson(int n, int m, int s, int t) {
	read(m);
	int ansFlow = 0, flow = 0;
	//Without scaling
	while ((flow = dfs(s, INF, 1, t)) > 0) {
		ansFlow += flow;
		curTime++;
	}
	//With scaling
	/*
	fornr (i, INF_LOG) 
		for (curTime++; (flow = dfs(s, INF, (1 << i), t)) > 0; curTime++)
			ansFlow += flow;
	*/
	return ansFlow;
}
\end{minted}
