Потоки:
\begin{table}[H]
\centering
\begin{tabular}{|l|l|}
  \hline
  Name                        & Asympthotic                 \\ \hline
  Ford-Fulkerson              & $O(|f| \cdot E)$            \\ \hline
  Ford-Fulkerson with scaling & $O(\log |f| \cdot E^2)$     \\ \hline
  Edmonds-Karp                & $O(V \cdot E^2)$            \\ \hline
  Dinic                       & $O(V^2\cdot E)$             \\ \hline
  Dinic with scaling          & $O(V \cdot E \cdot \log C)$ \\ \hline
  Dinic on bipartite graph    & $O(E \sqrt{V})$             \\ \hline
  Dinic on unit network       & $O(E \sqrt{E})$             \\ \hline
\end{tabular}
\end{table}

L---R потоки:

Есть граф с недостатками или избытками в каждой вершине. Создаем фиктивные исток и сток (из истока все ребра в избытки, из недостатков все ребра в сток).

Теперь пусть у нас есть L-R граф, для каждого ребра $e$ ($v \rightarrow u$) известны $L_e$ и $R_e$. Добавим в $v$ избыток $L_e$, в $u$ недостаток $L_e$, 
а пропускную способность сделаем $R_e - L_e$.

Получили решение задачи о LR-циркуляции.

Если у нас обычный граф с истоком и стоком, то добавляем бесконечное ребро из стока в сток и ищем циркуляцию.

Таким образом нашли удовлетворяющий условиям LR-поток. Если хотим максимальный поток, то на остаточной сети запускаем поиск максимального потока.

В новом графе в прямую сторону пропускная способность равна $R_e - f_e$, в обратную $f_e - L_e$.


MinCostCirculation:

Пока есть цикл отрицательного веса, запускаем алгоритм Карпа и пускаем максимальный поток по найденному циклу.

