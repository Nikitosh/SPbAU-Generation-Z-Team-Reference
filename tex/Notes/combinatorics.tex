Биномиальные коэффициенты:

Теорема Люка для биномиальных коэффициентов:
Хотим посчитать $C_n^k$, разложим в p-ичной системе счисления, 
$n = (n_0, n_1, \dots), k = (k_0, k_1, \dots)$.
$ans = C_{n_0} ^ {k_0} \cdot C_{n_1}^{k_1} \cdot \dots$

Способы вычисления $C_n^k$:
\begin{enumerate}
	\item
	$C_n^k = C_{n-1}^k + C_{n-1}^{k-1}$
	
	precalc: $O(n^2)$, query: $O(1)$.
	
	\item
	$C_n^k = \frac{n!}{k!(n - k)!}$, предподсчитываем факториалы
    
    precalc: $O(n \log n)$, query: $O(\log n)$
	
	\item
	Теорема Люка
    
    precalc: $O(p \log p)$, query: $O(log p)$.
	
	\item
	$C_n^k = C_n^{k-1} \cdot \frac{n - k + 1}{k}$

	\item
	$C_n^k = \frac{n!}{k!(n - k)!}$, для каждого факториала считаем степень вхождения и остаток
	
	precalc: $O(p \log p)$, query: $O(log p)$.

\end{enumerate}

$C_n^{\frac{n}{2}} = \frac{2^n}{\sqrt{\frac{\pi n}{2}}}$

