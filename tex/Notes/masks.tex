Считаем динамику по маскам за $O(2^n \cdot n)$ $f[mask] = sum$ по $submask$ $g[submask]$.

$dp[mask][i]$ --- значение динамики для маски $mask$, если младшие $i$ бит в ней зафиксированы (то есть мы не можем удалять оттуда).

Ответ в $dp[mask][0]$.

$dp[mask][len] = g[mask]$.
Если $i$-ый бит 0, то $dp[mask][i] = dp[mask][i + 1]$, иначе $dp[mask][i] = dp[mask][i + 1] + dp[mask ^ (1 << i)][i + 1]$.

Старший бит: предподсчет.
 
Младший бит: $x \& \sim (-x)$ 

Чтобы по степени двойки получить логарифм, можно воспользоваться тем, что все степени двойки имеют разный остаток по модулю $67$.

\begin{minted}[mathescape,tabsize=2,breaklines=true]{cpp}
for (int mask = 0; mask < (1 << n); mask++)
  submask : for (int s = mask; s; s = (s - 1) & mask)
  supmask : for (int s = mask; s < (1 << n); s = (s + 1) | mask)
\end{minted}

