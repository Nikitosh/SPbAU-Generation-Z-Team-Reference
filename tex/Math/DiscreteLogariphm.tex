\cpibegin
int discreteLogariphm(int a, int b, int mod) { //returns x: a^x = b (mod mod) or -1, if no such x exists
	int sq = sqrt(mod);
	int sq2 = mod / sq + (mod % sq ? 1 : 0);
	vector<pii> powers(sq2);
	forn (i, sq2)
		powers[i] = mp(power(a, (i + 1) * sq, mod), i + 1);
	sort(all(powers));
	forn (i, sq + 1) {
	 	int cur = power(a, i, mod);
	 	cur = (cur * 1ll * b) % mod;
	 	auto it = lower_bound(all(powers), mp(cur, 0));
	 	if (it != powers.end() && it->fs == cur)
	 		return it->sc * sq - i;
	}	
	return -1;
}
\end{minted}
