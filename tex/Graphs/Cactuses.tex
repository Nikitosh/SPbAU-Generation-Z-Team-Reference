\cpibegin
int used[MAX_N], inCycle[MAX_N], dp[MAX_N], inProcess[MAX_N];
vi g[MAX_N], sons[MAX_N], st, cycle;
set<pii> forbidden;
vector<vi> cycles;
int curCycle = 0;

void getCycles(int v, int p) {
	used[v] = 1;
	st.pb(v);
	for (int u : g[v])
		if (u != p && used[u] == 1) {
			cycle.clear();
			fornr (i, sz(st)) {
				cycle.pb(st[i]);
				inCycle[st[i]] = curCycle;
				if (st[i] == u)
					break;
			}		
			curCycle++;
			reverse(all(cycle));
			cycles.pb(cycle);
		}
		else if (u != p && !used[u])
			getCycles(u, v);
	st.pop_back();
	used[v] = 2;
}

bool isForbidden(int v, int u) {
	return forbidden.count(mp(v, u)) || forbidden.count(mp(u, v));
}

void dfs(int v, int p);

void calcTree(int v, int p) {
	used[v] = 1;
	for (int u : g[v])
		if (u != p && !isForbidden(v, u)) {
			dfs(u, v);
			//calc dp
		}
}

void calcCycle(int v, int p) {
	int c = inCycle[v];
	for (int u : cycles[c])
		inProcess[u] = 1;
	for (int u : cycles[c])
		for (int w : g[u])
			if (w != p && inCycle[w] != c)
				dfs(w, u), sons[u].pb(w);
	//calc dp on cactus
	for (int u : cycles[c])
		inProcess[u] = 0, used[u] = 1;
}

void dfs(int v, int p) {
	if (used[v])
		return;
	if (!inProcess[v] && inCycle[v] != -1) 
		calcCycle(v, p);
	else
		calcTree(v, p);
}	

int init(int n) {
	forn (i, n)
		inCycle[i] = -1;
	getCycles(0, -1);
	forn (i, n)
		used[i] = 0;
	dfs(0, -1);
	return dp[0];
}
\end{minted}
